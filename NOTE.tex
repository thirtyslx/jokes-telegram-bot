\documentclass{article}

\usepackage[russian]{babel}
\usepackage{relsize}
\usepackage{hyperref}

\usepackage[a4paper, total={6.3in, 8in}]{geometry}
\pagenumbering{gobble}

\author{}
\title{Jokes Telegram Bot \protect\\ Пояснительная записка \vspace{-4.3em}}
\date{}

\begin{document}

\setlength{\parskip}{\baselineskip}%
\relscale{1.5}

\maketitle

Проект является ботом для мессенджера Telegram.
Бот позволяет пользователю получать анекдоты выбранной или случайной категории в сообщениях в любое время, когда он захочет.

Бот выбирает анекдоты, распределенные по категориям из базы данных, собранной с сайта \underline{\href{https://anekdotov.net}{анекдотов.net}} с использованием черного списка категорий и фильтрацией по запрещенным словам.

Списки этих категорий и слов можно задать в переменных окружения.
Сборку можно запустить командой ``\verb+python -m bot.parser+'' или из телеграма через админ-панел, которую можно с помощью команды ``\verb+/admin+''.
Список админов можно задать в переменной окружения.

После сборки бд, пользователь может получить анекдот из случайной категории, либо выбрать конкретную и получать анекдоты оттуда.
Это можно сделать либо с помощью специальной кнопки, либо с помощью команды ``\verb+/change_category+''.

Стандартная клавиатура содержит динамическую кнопку для получения случайного анекдота, отображающая текущую категорию, и кнопка для смены категории.
Админ панель содержит кнопки для пересборки базы данных и для возвращения в обычное меню.

Сообщения с шутками содержат копирайт их владельца.

\end{document}